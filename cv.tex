\documentclass[12pt,a4paper,roman]{moderncv}       % possible options include font size ('10pt', '11pt' and '12pt'), paper size ('a4paper', 'letterpaper', 'a5paper', 'legalpaper', 'executivepaper' and 'landscape') and font family ('sans' and 'roman')

% moderncv themes
\moderncvstyle{classic}                            % style options are 'casual' (default), 'classic', 'banking', 'oldstyle' and 'fancy'
\moderncvcolor{burgundy}                           % color options 'black', 'blue' (default), 'burgundy', 'green', 'grey', 'orange', 'purple' and 'red'
%\renewcommand{\familydefault}{\sfdefault}         % to set the default font; use '\sfdefault' for the default sans serif font, '\rmdefault' for the default roman one, or any tex font name
%\nopagenumbers{}                                  % uncomment to suppress automatic page numbering for CVs longer than one page

% character encoding
\usepackage[utf8]{inputenc}                        % if you are not using xelatex ou lualatex, replace by the encoding you are using
\usepackage{xspace}
\usepackage[frenchb]{babel}

% adjust the page margins
\usepackage[scale=0.75]{geometry}
\usepackage[T1]{fontenc}
%\setlength{\hintscolumnwidth}{3cm}                % if you want to change the width of the column with the dates
%\setlength{\makecvheadnamewidth}{10cm}            % for the 'classic' style, if you want to force the width allocated to your name and avoid line breaks. be careful though, the length is normally calculated to avoid any overlap with your personal info; use this at your own typographical risks...

%\NoAutoSpaceBeforeFDP

% personal data
\name{Romain}{Warnan}
\title{Développeur senior \textit{full-stack}}
\address{11 bis rue des Missionnaires}{78000 Versailles}
\phone[mobile]{06~80~33~58~52}
\email{romain.warnan@gmail.com}
\social[github]{github.com/romain-warnan}
\photo[90pt][0pt]{pictures/romain-warnan.jpg}      % '64pt' is the height the picture must be resized to, 0.4pt is the thickness of the frame around it (put it to 0pt for no frame) and 'picture' is the name of the picture file
\quote{Some quote}                                 % optional, remove / comment the line if not wanted

\begin{document}

\makecvtitle

\section{Formation}\label{sec:formation}

\cventry{2006--2007}
{Qualification d'analyste développeur}
{Institut de gestion publique et du développement économique}
{Vincennes}
{}
{Conception de système d'information, et notions d'architecture informatique, de réseaux et de système d'exploitation}

\cventry{2004--2006}
{Concours d'attaché statisticien de l'Insee}
{École nationale de la statistique et de l'analyse de l'information}
{Rennes}
{}
{Mathématiques, statistiques, économie, informatique: algorithmie et langage Java}

\cventry{2002--2004}
{Classe préparatoire aux grandes écoles}
{Lycée Janson de Sailly}
{Paris}
{}
{Filière mathématiques, physique et sciences de l'ingénieur}

\section{Expérience professionnelle}\label{sec:expérience-professionnelle}
\cventry{depuis 2017}
{Responsable de domaine}
{Insee}
{Service national du dévelopement informatique de Paris}
{}
{Encadrement du groupe \og outils transverses \fg composé de 11 développeurs et assurant la maintenance d'applications transversales : gestions des annuaires, service de publipostage\ldots
\begin{itemize}%
  \item Encadrement de proximité ;
  \item Support et expertise aux autres développeurs ;
  \item Conception et animation de formation : Git et Spring MVC ;
  \item Animation du développement.
\end{itemize}}

\cventry{2014--2017}
{\textit{Lead-developer} : refonte du site \textit{insee.fr}}
{Insee}
{Centre national informatique de Paris}
{}
{Analyse et développements selon les méthodes Agiles dans le cadre d'un projet stratégique pour l'Insee : la refonte complète de son site de diffusion et de son \textit{back-office} :
\begin{itemize}%
  \item estimation des charges ;
  \item analyse, conception et développements des fonctionnalités les plus complexes ;
  \item prise en compte des contraintes : accessibilité, \textit{responsive design}, référencement,  sécurité et performances ;
  \item contrôle de la qualité des développements ;
  \item mise en production.
\end{itemize}%
Technologies utilisées :
\begin{itemize}%
  \item \textit{back-end} : Java avec Spring MVC, serveur Tomcat ;
  \item \textit{front-end} : JSP / jQuery, et React / Redux, Sass / Compass ;
  \item données : Postgres et SolR
\end{itemize}}

\cventry{2012--2014}
{\textit{Lead-developer} : refonte du site \textit{sirene.fr}}
{Insee}
{Centre national informatique de Paris}
{}
{Développements et encadrement des développeurs dans le cadre de la refonte du site de diffusion des données d'entreprises.
\newline Technologies utilisées :
\begin{itemize}%
  \item \textit{back-end} : Java avec Struts2, serveur Tomcat ;
  \item \textit{front-end} : JSP / jQuery / Bootstrap ;
  \item données : Postgres
  \item paiement en ligne : Paybox
\end{itemize}}

\cventry{2009--2012}
{Expert nomenclatures de professions}
{Insee}
{Direction des statistiques démographique et sociale}
{}
{}

\cventry{2007--2009}
{Responsable informatique des applications Java du recensement de la population}
{Insee}
{Centre national informatique de Paris}
{}
{Maintenance d'applications Java : \textit{batchs}, applications \textit{web} et clients lourds.}

\section{Langues étrangères}\label{sec:langues-etrangeres}
\cvitemwithcomment{Anglais}{Bon niveau}{Très bien lu, bien compris, assez bien parlé}

\section{Formation professionnelle}\label{sec:formation-professionelle}
\cvitem{Java}{Java avancé, Hibernate, Spring MVC, Struts2}
\cvitem{Javascript}{EcmaScript 6, React, Redux}
\cvitem{Web}{Responsive Web Design, Référencement, Accessibilité}
\cvitem{Données}{SolR, Big Data}
\cvitem{Productivité}{Git, Get Things Done, Manager de proximité}

\section{Compétences informatiques}\label{sec:competences-informatiques}
\cvdoubleitem{category 1}{XXX, YYY, ZZZ}{category 4}{XXX, YYY, ZZZ}
\cvdoubleitem{category 2}{XXX, YYY, ZZZ}{category 5}{XXX, YYY, ZZZ}
\cvdoubleitem{category 3}{XXX, YYY, ZZZ}{category 6}{XXX, YYY, ZZZ}

\end{document}