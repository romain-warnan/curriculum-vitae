\documentclass[12pt,a4paper,roman]{moderncv}        % possible options include font size ('10pt', '11pt' and '12pt'), paper size ('a4paper', 'letterpaper', 'a5paper', 'legalpaper', 'executivepaper' and 'landscape') and font family ('sans' and 'roman')

% moderncv themes
\moderncvstyle{classic}                              % style options are 'casual' (default), 'classic', 'banking', 'oldstyle' and 'fancy'
\moderncvcolor{burgundy}                           % color options 'black', 'blue' (default), 'burgundy', 'green', 'grey', 'orange', 'purple' and 'red'
%\renewcommand{\familydefault}{\sfdefault}         % to set the default font; use '\sfdefault' for the default sans serif font, '\rmdefault' for the default roman one, or any tex font name
%\nopagenumbers{}                                  % uncomment to suppress automatic page numbering for CVs longer than one page

% character encoding
\usepackage[utf8]{inputenc}                        % if you are not using xelatex ou lualatex, replace by the encoding you are using
\usepackage{xspace}
\usepackage[frenchb]{babel}

% adjust the page margins
\usepackage[scale=0.75]{geometry}
\usepackage[T1]{fontenc}
%\setlength{\hintscolumnwidth}{3cm}                % if you want to change the width of the column with the dates
%\setlength{\makecvheadnamewidth}{10cm}            % for the 'classic' style, if you want to force the width allocated to your name and avoid line breaks. be careful though, the length is normally calculated to avoid any overlap with your personal info; use this at your own typographical risks...

%\NoAutoSpaceBeforeFDP

% personal data
\name{Romain}{Warnan}
\title{Développeur senior \textit{full-stack}}
\address{11 bis rue des Missionnaires}{78000 Versailles}
\phone[mobile]{06~80~33~58~52}
\email{romain.warnan@gmail.com}
\social[github]{github.com/romain-warnan}
\photo[90pt][0pt]{pictures/romain-warnan.jpg}      % '64pt' is the height the picture must be resized to, 0.4pt is the thickness of the frame around it (put it to 0pt for no frame) and 'picture' is the name of the picture file
\quote{Some quote}                                 % optional, remove / comment the line if not wanted

\begin{document}

\makecvtitle

\section{Formation}\label{sec:formation}

\cventry{2006--2007}
{Qualification d'analyste développeur}
{Institut de gestion publique et du développement économique}
{Vincennes}
{}
{Conception de système d'information, et notions d'architecture informatique, de réseaux et de système d'exploitation}

\cventry{2004--2006}
{Concours d'attaché statisticien de l'Insee}
{École nationale de la statistique et de l'analyse de l'information}
{Rennes}
{}
{Mathématiques, statistiques, économie, informatique: algorithmie et langage Java}

\cventry{2002--2004}
{Classe préparatoire aux grandes écoles}
{Lycée Janson de Sailly}
{Paris}
{}
{Filière mathématiques, physique et sciences de l'ingénieur}

\section{Expérience professionnelle}\label{sec:expérience-professionnelle}
\cventry{depuis 2017}
{Responsable de domaine}
{Insee}
{Service national du dévelopement informatique de Paris}
{}
{Encadrement du groupe \og outils transverses \fg composé de 11 développeurs et assurant la maintenance d'applications transversales : gestions des annuaires, service de publipostage\ldots
\begin{itemize}%
  \item Encadrement de proximité ;
  \item Support et expertise aux autres développeurs ;
  \item Conception et animation de formation : Git et Spring MVC ;
  \item Animation du développement.
\end{itemize}}

\cventry{2014--2017}
{\textit{Lead-developer} : refonte du site \textit{insee.fr}}
{Insee}
{Centre national informatique de Paris}
{}
{Analyse et développements selon les méthodes Agiles dans le cadre d'un projet stratégique pour l'Insee : la refonte complète de son site de diffusion et de son \textit{back-office} :
\begin{itemize}%
  \item estimation des charges ;
  \item analyse, conception et développements des fonctionnalités les plus complexes ;
  \item prise en compte des contraintes : accessibilité, \textit{responsive design}, référencement,  sécurité et performances ;
  \item contrôle de la qualité des développements ;
  \item mise en production.
\end{itemize}%
Technologies utilisées :
\begin{itemize}%
  \item \textit{back-end} : Java avec Spring MVC, serveur Tomcat ;
  \item \textit{front-end} : JSP / jQuery, et React / Redux, Sass / Compass ;
  \item données : Postgres et SolR
\end{itemize}}

\cventry{2012--2014}
{\textit{Lead-developer} : refonte du site \textit{sirene.fr}}
{Insee}
{Centre national informatique de Paris}
{}
{Développements et encadrement des développeurs dans le cadre de la refonte du site de diffusion des données d'entreprises.
\newline Technologies utilisées :
\begin{itemize}%
  \item \textit{back-end} : Java avec Struts2, serveur Tomcat ;
  \item \textit{front-end} : JSP / jQuery / Bootstrap ;
  \item données : Postgres
  \item paiement en ligne : Paybox
\end{itemize}}

\cventry{2009--2012}
{Expert nomenclatures de professions}
{Insee}
{Direction des statistiques démographique et sociale}
{}
{}

\cventry{2007--2009}
{Responsable informatique des applications Java du recensement de la population}
{Insee}
{Centre national informatique de Paris}
{}
{Maintenance d'applications Java : \textit{batchs}, applications \textit{web} et clients lourds.}

\section{Langues étrangères}\label{sec:langues-etrangeres}
\cvitemwithcomment{Anglais}{Bon niveau}{Très bien lu, bien compris, assez bien parlé}

\section{Formation professionnelle}\label{sec:formation-professionelle}
\cvitem{Java}{Java avancé, Hibernate, Spring MVC, Struts2}
\cvitem{Javascript}{EcmaScript 6, React, Redux}
\cvitem{Web}{Responsive Web Design, Référencement, Accessibilité}
\cvitem{Données}{SolR, Big Data}
\cvitem{Productivité}{Git, Get Things Done, Manager de proximité}

\section{Computer skills}
\cvdoubleitem{category 1}{XXX, YYY, ZZZ}{category 4}{XXX, YYY, ZZZ}
\cvdoubleitem{category 2}{XXX, YYY, ZZZ}{category 5}{XXX, YYY, ZZZ}
\cvdoubleitem{category 3}{XXX, YYY, ZZZ}{category 6}{XXX, YYY, ZZZ}

\section{Interests}
\cvitem{hobby 1}{Description}
\cvitem{hobby 2}{Description}
\cvitem{hobby 3}{Description}

\section{Extra 1}
\cvlistitem{Item 1}
\cvlistitem{Item 2}
\cvlistitem{Item 3. This item is particularly long and therefore normally spans over several lines. Did you notice the indentation when the line wraps?}

\section{Extra 2}
\cvlistdoubleitem{Item 1}{Item 4}
\cvlistdoubleitem{Item 2}{Item 5\cite{book1}}
\cvlistdoubleitem{Item 3}{Item 6. Like item 3 in the single column list before, this item is particularly long to wrap over several lines.}

\section{References}
\begin{cvcolumns}
  \cvcolumn{Category 1}{\begin{itemize}\item Person 1\item Person 2\item Person 3\end{itemize}}
  \cvcolumn{Category 2}{Amongst others:\begin{itemize}\item Person 1, and\item Person 2\end{itemize}(more upon request)}
  \cvcolumn[0.5]{All the rest \& some more}{\textit{That} person, and \textbf{those} also (all available upon request).}
\end{cvcolumns}

% Publications from a BibTeX file without multibib
%  for numerical labels: \renewcommand{\bibliographyitemlabel}{\@biblabel{\arabic{enumiv}}}% CONSIDER MERGING WITH PREAMBLE PART
%  to redefine the heading string ("Publications"): \renewcommand{\refname}{Articles}
\nocite{*}
\bibliographystyle{plain}
\bibliography{publications}                        % 'publications' is the name of a BibTeX file

% Publications from a BibTeX file using the multibib package
%\section{Publications}
%\nocitebook{book1,book2}
%\bibliographystylebook{plain}
%\bibliographybook{publications}                   % 'publications' is the name of a BibTeX file
%\nocitemisc{misc1,misc2,misc3}
%\bibliographystylemisc{plain}
%\bibliographymisc{publications}                   % 'publications' is the name of a BibTeX file

\clearpage
%-----       letter       ---------------------------------------------------------
% recipient data
\recipient{Company Recruitment team}{Company, Inc.\\123 somestreet\\some city}
\date{January 01, 1984}
\opening{Dear Sir or Madam,}
\closing{Yours faithfully,}
\enclosure[Attached]{curriculum vit\ae{}}          % use an optional argument to use a string other than "Enclosure", or redefine \enclname
\makelettertitle

Lorem ipsum dolor sit amet, consectetur adipiscing elit. Duis ullamcorper neque sit amet lectus facilisis sed luctus nisl iaculis. Vivamus at neque arcu, sed tempor quam. Curabitur pharetra tincidunt tincidunt. Morbi volutpat feugiat mauris, quis tempor neque vehicula volutpat. Duis tristique justo vel massa fermentum accumsan. Mauris ante elit, feugiat vestibulum tempor eget, eleifend ac ipsum. Donec scelerisque lobortis ipsum eu vestibulum. Pellentesque vel massa at felis accumsan rhoncus.

Suspendisse commodo, massa eu congue tincidunt, elit mauris pellentesque orci, cursus tempor odio nisl euismod augue. Aliquam adipiscing nibh ut odio sodales et pulvinar tortor laoreet. Mauris a accumsan ligula. Class aptent taciti sociosqu ad litora torquent per conubia nostra, per inceptos himenaeos. Suspendisse vulputate sem vehicula ipsum varius nec tempus dui dapibus. Phasellus et est urna, ut auctor erat. Sed tincidunt odio id odio aliquam mattis. Donec sapien nulla, feugiat eget adipiscing sit amet, lacinia ut dolor. Phasellus tincidunt, leo a fringilla consectetur, felis diam aliquam urna, vitae aliquet lectus orci nec velit. Vivamus dapibus varius blandit.

Duis sit amet magna ante, at sodales diam. Aenean consectetur porta risus et sagittis. Ut interdum, enim varius pellentesque tincidunt, magna libero sodales tortor, ut fermentum nunc metus a ante. Vivamus odio leo, tincidunt eu luctus ut, sollicitudin sit amet metus. Nunc sed orci lectus. Ut sodales magna sed velit volutpat sit amet pulvinar diam venenatis.

Albert Einstein discovered that $e=mc^2$ in 1905.

\[ e=\lim_{n \to \infty} \left(1+\frac{1}{n}\right)^n \]

\makeletterclosing

%\clearpage\end{CJK*}                              % if you are typesetting your resume in Chinese using CJK; the \clearpage is required for fancyhdr to work correctly with CJK, though it kills the page numbering by making \lastpage undefined
\end{document}


%% end of file `template.tex'.
